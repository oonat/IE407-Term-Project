\begin{table}[H]
	\centering
	\begin{tabular}{| c | c |}
		\hline
		Initial Profit & 6945.73351921896 \\ \hline
		New Profit     & 6943.00439273943  \\ \hline
		Profit Change  & -2.72912647953 \\ \hline
		Decision Variables  & Check \ref{q7_res}  \\ \hline
	\end{tabular}
	\caption{Q7 Comparison Table}
\end{table}
In this section, we analysed how the solution found in \ref{q2} changes when the restriction requiring Product 3 and Product 8 to be placed on different shelves is removed and the restriction requiring these two products to be assorted together and placed to the same shelf is added.

As a result of this change in constraints, we should see a change in profits probably. Nevertheless, this may not always happen. In the case of the products are never put in the shelves, or the constraint does not really constrain our optimal value, we should see that profit does not change. Otherwise, we still cannot say that this will change profit positively or negatively. The reason is we are removing one constraint, and adding a new constraint. Generally, what we can say is if we are removing a constraint, the profit will be at least as good as before, and vice versa. However, here two things happening, we are removing a constraint, and at the same time adding a new constraint. Before solving the model, it is almost impossible to say whether the profit will increase or decrease.

We removed the constraint in the model and saw that profit decreased a little bit. From that, we can conclude that it is better if we put product 3 and 8 separately.\\
\\
\underline{\textbf{Changes in the excel model:}}\\
We deleted the constrain that were making sure products 3 and 8 were not in the same shelf and added below constrain to make sure they were assorted together(if any one of them were assorted the other was also assorted) and were in the same shelf:\\
\begin{equation*}
\begin{split}
(\sum^{4}_{j=1}X_{3jk})-(\sum^{4}_{j=1}X_{8jk})=0 \quad \forall k \in K \\
\end{split}
\end{equation*}