\subsection{Data Set for Q1,Q2}\label{data-set}
\begin{table}[H]
\centering
\caption{ Sets of product pairs }
\begin{tabular}{ c c }
$I^1$ : & (2, 5) , (3, 8), (16, 20)\\ 
$I^2$ : & (1, 12), (3, 8), (9, 15), (16, 20) \\  
\end{tabular}
\end{table}


\begin{table}[H]
\centering
\caption{ Product based data }
\begin{tabular}{ c c c c c c c c }
Product number& $\pi_i$& $b_i$& $dp_i$& $d_i$& $\beta_i$& $s^{l}_i$&$s^{U}_i$\\
1&15&10&7&4&0.5&2&15\\
2&8&9&6&10&0.2&1&19\\
3&12&5&10&10&0.3&1&23\\
4&6&7&7&7&0.2&1&9\\
5&11&9&9&5&0.9&1&16\\
6&14&6&8&2&0.4&1&21\\
7&14&9&6&1&0.5&1&11\\
8&6&5&9&6&0.3&1&18\\
9&5&9&9&7&0.8&2&11\\
10&11&10&9&3&0.8&1&11\\
11&12&7&5&4&0.1&2&17\\
12&8&5&6&7&0.8&2&22\\
13&11&7&6&2&0.1&1&12\\
14&13&8&9&9&0.8&3&12\\
15&7&9&8&11&0.8&3&19\\
16&14&23&5&2&0.1&2&16\\
17&9&25&6&9&0.1&1&10\\
18&10&17&8&1&0.8&2&16\\
19&13&15&9&4&0.2&2&20\\
20&5&23&6&2&0.6&3&19\\
21&11&19&8&6&0.6&2&24\\
22&11&19&9&6&0.4&1&16\\
23&7&16&7&8&0.5&2&13\\
24&10&14&5&2&0.8&1&16\\
25&13&16&10&4&0.9&2&14\\
 \end{tabular}
\end{table}

\begin{table}[H]
\centering
\caption{ Shelf based data }
\begin{tabular}{ c c c c c c c c }
Shelf number&$w_k$&$ds_k$&$\gamma k$\\
1&50&34&0.25\\
2&65&30&0.60\\
3&80&26&1\\
4&95&27&0.60\\
5&110&29&0.25\\
 \end{tabular}
\end{table}
\subsection{Rules}\label{rules}
1- \textbf{If a product is selected in the assortment, than all facings for the product must be placed on the same shelf.}$\sqrt{}$\\
2- \textbf{The manager does not want to allocate more than four facings for a
product.} $\sqrt{}$\\
3- \textbf{If a product is selected in the assortment, then a minimum shelf inventory amount must be placed on the shelves. Similarly, for each product there is an upper bound on the shelf-inventory.}$\sqrt{}$\\
4- \textbf{If a product is selected in the assortment, lower and upper bounds on its facing number are calculated by using shelf depth, product depth and lower and upper bounds on the shelf-inventory.}$\sqrt{}$\\
5- \textit{Each product provides a certain profit per unit sold.}\\
6- \textbf{Each shelf has a certain width and the total width of the facings placed in the shelf cannot exceed its width.}$\sqrt{}$\\
7- \textbf{For some pairs of products, there is a restriction that if one is included in the assortment, the other product must also be included.}$\sqrt{}$\\
8- \textbf{For some pairs of products, there is a restriction that they cannot be on the same shelf}$\sqrt{}$\\
\subsection{Opensolver Results}\label{opensolver_res}
 Objective function results for the scenarios are already given and discussed in the \ref{p_body} section. In this section, the tables for the decision variable values calculated by opensolver for the corresponding objective function results are given. \\ \\
 \[
 X_{ijk} =
 \begin{cases}
 1, & if \ i \ is \ allocated \ j \ times \ (i.e. \ has \ j \ facings) \ in \  k \\
 0, & otherwise \\
 \end{cases}
 \]
 
 Tables given below represent $X_{i1k}$, $X_{i2k}$, $X_{i3k}$, $X_{i4k}$ from left to right, respectively. For each table,
\begin{itemize}
	\item  Column k represents kth shelf,
	\item  Row i represents ith product
\end{itemize}
For instance, if the value in the table 3, row 4 and column 2 equals to 1, that denotes that to obtain the optimum value of the objective function, product 4 would be placed in 2nd shelf with 3 allocations (facings).

\subsubsection{Q2 Result}\label{q2_res}
\begin{figure}[H]
	\caption{Q2 Result} 
	\makebox[\textwidth]{\includegraphics[scale = 0.43]{q2_exc}}
\end{figure}
\subsubsection{Q3 Result}\label{q3_res}
\begin{figure}[H]
	\caption{Q3 Result} 
	\makebox[\textwidth]{\includegraphics[scale = 0.43]{q3_exc}}
\end{figure}
\begin{figure}[H]
	\caption{Q3 Result cont'd} 
	\makebox[\textwidth]{\includegraphics[scale = 0.43]{q3_exc_cont}}
\end{figure}
\subsubsection{Q4 Result}\label{q4_res}
\begin{figure}[H]
	\caption{Q4 Result} 
	\makebox[\textwidth]{\includegraphics[scale = 0.43]{q4_exc}}
\end{figure}
\subsubsection{Q5 Result}\label{q5_res}
\begin{figure}[H]
	\caption{Q5 Result} 
	\makebox[\textwidth]{\includegraphics[scale = 0.43]{q5_exc}}
\end{figure}
\subsubsection{Q6 Result - 5cm increase}\label{q6_res_1}
\begin{figure}[H]
	\caption{Q6 Result after 5cm increase} 
	\makebox[\textwidth]{\includegraphics[scale = 0.43]{q6_exc_1}}
\end{figure}
\subsubsection{Q6 Result - 10cm increase}\label{q6_res_2}
\begin{figure}[H]
	\caption{Q6 Result after 10cm increase} 
	\makebox[\textwidth]{\includegraphics[scale = 0.45]{q6_exc_2}}
\end{figure}
\subsubsection{Q7 Result}\label{q7_res}
\begin{figure}[H]
	\caption{Q7 Result} 
	\makebox[\textwidth]{\includegraphics[scale = 0.43]{q7_exc}}
\end{figure}
