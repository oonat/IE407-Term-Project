Initially, we approached the problem by defining two decision variables: \\ \\
$X_{ik}:$ Will the product i be on shelf k. Such that,\\
\[
X_{ik} =
\begin{cases}
1, & if \ i \ is \ allocated \ in \  k \\
0, & otherwise \\
\end{cases}
\]
$f_{ik}$: Number of allocated facings for product i on shelf k.\\ \\
With these decision variables, the objective function (since our aim is maximizing the profit) became:
\begin{equation*}
\begin{split}
max \sum^{|I|}_{i=1}\sum^{|K|}_{k=1}\pi_i \gamma_k d_i(f_{ik} * b_i)^{\beta_i} \\
\end{split}
\end{equation*}
As can be confirmed, if we choose $f_{ik}$ as a decision variable then the given objective function makes the model non-linear. Since the "opensolver" is not good at solving non-linear problems, we decided to modify the initial model to obtain a linear model. \\ \\
The proposed model \ref{p_model} has been built considering (with the help of) Rule 2. Since for a product we can not allocate more than 4 facings (i.e. $f_{ik} \in \{1,2,3,4\}$, if the product is chosen for the assortment), defining a seperate decision variable for facings was unnecessary. Rather we defined new decision variables instead of the previous ones, \\ \\
$X_{ijk}:$ Will the product i have j allocations (facings) on shelf k. Such that,\\
\[
X_{ijk} =
\begin{cases}
1, & if \ i \ is \ allocated \ j \ times \ in \  k \\
0, & otherwise \\
\end{cases}
\]
\\
where $j \in \{1,2,3,4\}$. With this new approach, we have achieved to develop a linear model which can be seen in \ref{p_model}.
