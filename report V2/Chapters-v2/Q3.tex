Objective function result was found as: 37906,15197\\
\\
In our model we are using 4*ProductCount*ShelfCount decision variables\\
(500 for smaller, 2000 for bigger data set)

While the lesser data set worked in an instant larger data set took 34 seconds to finish.\\
This is because number of decision variables increases, which in turn increases\\
the time it takes to solve relaxation of a problem and number of problems.\\
\\
For a problem like this the worst case is complete enumeration of decision variables.\\
And all of our decision variables are binary which means normally the worst case\\
we would have needed to solve $2^{500}$ for lesser data set and $2^{2000}$ for bigger data set.\\
but because of our first constraint for any i and k values we would have 5 cases which are:\\
case1: 0 0 0 0\\
case2: 1 0 0 0\\
case3: 0 1 0 0\\
case4: 0 0 1 0\\
case5: 0 0 0 1\\
That means we have only 5 different cases for every 4 decision variables instead of $2^{4}$
Because of that in the case of complete enumeration we would have\\
$5^{100}$ problems for smaller data set and $5^{400}$ for bigger data set.\\
(these numbers were only calculated using first constrain. By adding others we can reduce them more.)\\
so as we can see for the complete enumeration our problem number increases exponentially.\\
Because of that solution time will also increase exponentially.\\
\\
Also since we are using more constraints and decision variables, the size of the matrixes\\
we are going to use to solve an LP relaxation of a problem, will increase in size.\\
\\
Size of B matrix is dependent on constrain count.\\
Size of N matrix is dependent on variable count and size of B.\\
Lets calculate sizes of these matrixes for each data set:\\
\\
1- Smaller data set:\\
400 decision variables\\
25 constrains from rule 1st group of constraints. which also adds 25 slack variables($<$=)\\
25 constrains from rule 2nd group of constraints. which also adds 25 excess variables($>$=)\\
25 constrains from rule 3rd group of constraints. which also adds 25 slack variables($<$=)\\
5 constrains from rule 4th group of constraints. which also adds 5 slack variables($<$=)\\
4 constrains from rule 5th group of constraints. (=)\\
15 constrains from rule 6th group of constraints. which also adds 15 slack variables($<$=)\\
so we have 99 constrains\\
and 495 decision variables\\
495-99=396\\
\\
which means size of our Matrices are:\\
B=99*99\\
N=99*396\\
$x_B$=99*1\\
$x_N$=396*1\\
$c_B$=99*1\\
$c_N$=396*1\\
b=99*1\\
$B^{-1}$=99*99\\
\\
2- Larger data set:\\
2000 decision variables\\
100 constrains from rule 1st group of constraints. which also adds 100 slack variables($<$=)\\
100 constrains from rule 2nd group of constraints. which also adds 100 excess variables($>$=)\\
100 constrains from rule 3rd group of constraints. which also adds 100 slack variables($<$=)\\
5 constrains from rule 4th group of constraints. which also adds 5 slack variables($<$=)\\
13 constrains from rule 5th group of constraints. (=)\\
60 constrains from rule 6th group of constraints. which also adds 60 slack variables($<$=)\\
so we have 378 constrains\\
and 2365 decision variables\\
2365-378=1987\\
\\
which means size of our Matrices are:\\
B=378*378\\
N=378*1987\\
$x_B$=378*1\\
$x_N$=1987*1\\
$c_B$=378*1\\
$c_N$=1987*1\\
b=378*1\\
$B^{-1}$=378*378\\
\\
Note: and those numbers are calculated while ignoring artificial variables\\.
\\
Since we are going to use simplex method we would need to make matrix multiplications.\\
Which are the most costly operations in big data sets.\\
(if you are making opearation: AxB for A matrix being a*b and B matrix b*c you would need to do a*b*c multiplications)\\ 
\\
Lets just calculate biggest operation we are going to need to do:\\
which is $c_B^{T}*B^{-1}*N$ (its actually: $c_N^{T}-c_B^{T}*B^{-1}*N$ but matrix multiplication is much slower than matrix differention so it doesnt make almost any difference)
\\
For first data set the number of operations this multiplication will take is:\\
(1*99*99)*(1*99*396)=384238404 operations\\
For first data set the number of operations this multiplication will take is:\\
(1*378*378)*(1*378*1987)=107318172024 operations\\
\\ 
283910508/3881196=279.30100403\\
\\
so as we can easily see, the time it takes to calculate this operation has increased\\
more than 275 times. But since difference between other operations wont be as big as this\\
Our operation time will be a little less then this number.
which is a great increase but nowhere near the increase on number of problems.\\
\\
So in the end solution time of our problem will grow exponentially because of the increase in problem count.\\

