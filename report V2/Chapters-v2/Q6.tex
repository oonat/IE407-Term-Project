In this section, we analysed how the profit found in \ref{q2} changes and what are the basic variables when two seperate cases happen:
\begin{itemize}
	\item the width of Shelf 5 ($w_5$) is increased by 5 cm
	\item the width of Shelf 5 ($w_5$) is increased by 10 cm
\end{itemize}

Firstly, for both cases, we are relaxing constraints. Therefore, we can say that our profit will be a little higher, or will be the same.

For the first case, we see that the new profit is x. As a result of increasing shelf width, we gain a y profit increase.

For the second case, we are relaxing constraints a bit more. As a result of increasing shelf width even more, our new profit is z, even t more than the first case, as we expected.

What we can say about our basic variables is \dots

Lastly, in classes, we see that as we relax the constraints more, the optimized value will be worse, or the same. And here, we especially see how the constraint changes are affecting the value in practice.