To summarize, in this project we tried to solve Assortment Selection and Shelf Space Allocation Problem. We have designed a mathematical model that fits to given constraints and achieved to obtain optimized objective function values and corresponding decision variables. As a result of our analysis on the dataset given in \ref{data-set}, we have found the objective function value as \textbf{6945.73351921896}. Furthermore, we got the highest profit of \textbf{7165.30316863258} among the given scenarios when we added a new shelf (\ref{q4_reference}). In this scenario, we have achieved \textbf{+219.569649414} profit increase. \\

In conclusion, the employees should place the most profitable products on shelves that has highest shelf effect on demand. When adding a shelf that has shelf effect on demand value($\gamma_k$) of lets say A, the employees should move some of the most profitable products from shelves that has lower shelf effect on demand value($\gamma_k$) then A to the new shelf. Also when removing a shelf we should consider to move the most profitable products on that shelf to other shelves even if the shelf we are removing has the lowest shelf effect on demand value($\gamma_k$). Lastly if a product is going to be assorted the employees should choose how many of facings a product will have based on its $\beta_i$ value(for most part) and other coefficients($\pi_i,b_i,d_i$).